%
\definecolor{muxpurple}{RGB}{210,140,210}%
\definecolor{entryred}{RGB}{255,50,50}%
\definecolor{exitgreen}{RGB}{70,200,80}%
\definecolor{slightblue}{RGB}{230,230,255}%
%
\tikzstyle{startstop}=[%
	inner sep=\inndist,%
	rectangle,%
	rounded corners,%
	minimum width=3cm,%
	minimum height=1cm,%
	text centered,%
	align=center,%
	draw=black,%
	fill=red!30%
]%
\tikzstyle{io}=[%
	trapezium,%
	trapezium left angle=70,%
	trapezium right angle=110,%
	minimum width=2cm,%
	minimum height=1cm,%
	inner ysep=\inndist,%
	inner xsep=0.15*\inndist,%
	text centered,%
	align=center,%
	draw=black,%
	fill=blue!30%
]%
\tikzstyle{interthread}=[%
	trapezium,%
	trapezium left angle=-70,%
	trapezium right angle=-110,%
	minimum width=2cm,%
	minimum height=1cm,%
	inner ysep=\inndist,%
	inner xsep=0.15*\inndist,%
	text centered,%
	align=center,%
	draw=black,%
	fill=cyan!30!white!90!black%
]%
\tikzstyle{processsplitl}=[%
	minimum width=3cm,%
	%minimum height=1cm,%
	align=left,%
	inner sep=0.5*\inndist,%
	draw=black,%
	rectangle split,%
	rectangle split parts=#1,%
	rectangle split part fill={orange!30,orange!30},%
    rectangle split every empty part={},% delete existing height, depth and width
	rectangle split empty part height=0.1ex,%
	rectangle split empty part depth=0ex,%
	rectangle split part align=left,%either: top, center, base, bottom, left, right
%			You could do something like this, to do differing aligns for multiple parts:
%				rectangle split part align={center, left, left, left},%
% 	shade,%
% 	shading=axis,%
% 	top color=white,%
% 	bottom color=processblue,%
% 	shading angle=45,%
]%
\tikzstyle{processsplit}=[%
	processsplitl=#1,%
	rectangle split part align=center,%
	text centered,%
	align=center,%
]%
\tikzstyle{process}=[%
	rectangle,%
	minimum width=2cm,%
	minimum height=1cm,%
	text centered,%
	align=center,%
	draw=black,%
	fill=orange!30%
]%
\tikzstyle{decision}=[%
	diamond,%
	aspect=3,%
	minimum width=3cm,%
	minimum height=1cm,%
	text centered,%
	align=center,%
	draw=black,%
	fill=green!30%
]%
\tikzstyle{decision2}=[%
	decision,%
	diamondcut,%
	aspect=1,%
	minimum width=0cm,%
	minimum height=0cm,%
	minimum size=0cm,%
	inner sep=0.25ex,%
]%
\tikzstyle{mux}=[%
	trapezium,%
	trapezium left angle=110,%
	trapezium right angle=110,%
	minimum width=3cm,%
	minimum height=1cm,%
	inner sep=\inndist,%
% 	inner sep=0pt,%
	text centered,%
	align=center,%
	draw=black,%
	fill=muxpurple%
]%
\newcommand{\muxwidth}[1]{#1*3cm+2*0.36cm}%
\tikzstyle{mux2}=[%
	trapezium,%
	trapezium left angle=-70,%
	trapezium right angle=110,%
	trapezium stretches=true,%
	trapezium stretches body=true,%
	minimum size=0.0cm,%
	minimum width=#1*3cm+#1*\pgflinewidth+0.6cm,%
	minimum height=1.0cm,%
	inner sep=\inndist,%
% 	inner xsep=0.0cm,%
% 	outer sep=0.0cm,%
% 	text height=1.0cm,%
% 	text width=6.0cm,%
	text centered,%
	align=center,%
	draw=black,%
	fill=muxpurple%
]%
\tikzstyle{muxwahl}=[%
	regular polygon,%
	regular polygon sides=3,%
	shape border rotate=180,%
	yscale=0.35,%
	minimum width=3.5cm,%
	minimum height=0.1cm,%
	inner sep=0.0cm,%
	text centered,%
	align=center,%
	draw=black,%
	fill=muxpurple%
]%
\tikzstyle{muxwahl2}=[%
	trapezium,%
	trapezium left angle=-45,%
	trapezium right angle=-45,%
	minimum width=3cm,%
% 	text width=3cm,%
	minimum height=1cm,%
	inner sep=\inndist,%
% 	inner sep=0pt,%
	text centered,%
	align=center,%
	draw=black,%
	fill=muxpurple%
]%
\tikzstyle{note}=[%
	rectangle,%
	minimum width=2cm,%
	minimum height=1cm,%
	text centered,%
	align=center,%
	draw=black,%
	dashed,%
	rounded corners,%
]%
\tikzstyle{cleartext}=[%
	solid,%
	align=center,%
	inner sep=0pt,%
	minimum width=0pt,%
	minimum height=0pt,%
]%
%
%
%-----------------------------------------------------------
% For Finite State Machines (FSM)
%-----------------------------------------------------------
%========-------------------------------------------========
%========		First some specials					========
%========-------------------------------------------========
% They are for "`nice"' double border nodes
\tikzset{doubleA/.style = {matrix of nodes,%
    draw, inner sep=1mm,%
    nodes = {rectangle, draw, inner sep=.3333em}}}%
%
\tikzset{doubleB/.style = {circle, draw, %
    append after command={%
        \pgfextra{\node[fit=(\tikzlastnode),%
        	circle,%
        	draw,%
        	minimum size=0pt,%
        	minimum height=0pt,%
        	minimum width=0pt,%
        	inner sep=0pt]%
        	(\tikzlastnode-out){};%
        }%
    }%
    }}%
% Example for use of doubleA and doubleB:
%	Google Drive\Bibliothek\Latex\tikz\Beispiele,Templates,usw\doubleAdoubleB-Bsp.tex
% ========-------------------------------------------========
\tikzstyle{fsmstart2}=[%
	circle,%
	draw=black,%
	line width=1.5pt,%
	double=white,%
	double distance=0.5ex,%
	inner sep=1ex,%
	text centered,%
	align=center,%
]%
% Alternative. Gives both the Anchors on the outer and the inner border
\tikzstyle{fsmstart}=[%
	doubleB,%
	inner sep=1.0ex,%
]%
\tikzstyle{fsmfinish}=[%
	circle,%
	draw=black,%
	inner sep=1.5ex,%
]%
\tikzstyle{fsmstate}=[%
	draw,%
	shape=ellipse,%
	anchor=base,%
	inner sep=1.0ex,%
	text centered,%
	align=center,%
]%
%========-------------------------------------------========
%========		UML State Machine					========
%========-------------------------------------------========
% Usage:
% - - - - - - -
%        \node(S1)%
%            [UMLSM_state]%
%            {node
%            \nodepart{two} a};%
%        \node(S2)%
%            [UMLSM_state,anchor=north west]%
%            at($(S1.north east)+(\nodedist,0)$)
%            {node 2};%
%        \path[UMLSM_arrow](S1.east)--(S2.west);%
% - - - - - - -
\tikzstyle{UMLSM_start}=[%
    circle,draw=black,fill=black,%
    inner sep=0pt,minimum size=1.25ex,%
]%
\tikzstyle{UMLSM_state}=[%
    minimum width=6em,%
    %minimum height=1cm,%
    inner ysep=0.5*\inndist,%
    inner xsep=1.0*\inndist,%
    draw=black,%
    align=left,% To enable linebreak
    rounded corners=1.5ex,%
    rectangle split,%
    rectangle split parts=2,%
    rectangle split part fill={gray!40,white},%
    rectangle split every empty part={},% delete existing height, depth and width
    rectangle split empty part height=1.75ex,%
    rectangle split empty part depth=0ex,%
    rectangle split part align={center,left},%either: top, center, base, bottom, left, right
]%
\tikzstyle{UMLSM_arrow}=[%
    draw,%
    thick,%
    line join=round,%
    rounded corners=1.3ex,%
    >=latex,% latex, stealth
    ->,%
]%
%
%
%
%-----------------------------------------------------------
% Generic Data-Flowing
%-----------------------------------------------------------
%========-------------------------------------------========
%========		Definitions					========
%========-------------------------------------------========
\let\procwidth\undefined%
\newlength\procwidth%
\settowidth{\procwidth}{\textbf{Forwarding Path}}%
\setlength\procwidth{\procwidth+\inndist+\inndist+1em}%
\setlength\procwidth{6em}%
%
\let\procheight\undefined%
\newlength\procheight%
%
\setlength\procheight{1.75\baselineskip}%
%
\newcommand\roundcorneramount{1.25ex}%
%========-------------------------------------------========
%========		Shapes					========
%========-------------------------------------------========
\tikzstyle{process}=[%
	rectangle,%
	rounded corners=\roundcorneramount,%
	minimum width=\procwidth,%
	minimum height=\procheight,%
	text centered,%
	align=center,%
	inner sep=0pt,%\inndist,
	draw=black%
	%fill=gray!30%
]%
\tikzstyle{threadinvisible}=[%
	rectangle,%
	rounded corners=\roundcorneramount,%
	minimum width=3cm,%
	minimum height=1cm,%
	text centered,%
	align=center,%
% 	draw=black,%
% 	fill=red!30%
]%
%========-------------------------------------------========
%========		Text For Arrows					========
%========-------------------------------------------========
\tikzstyle{input}=[%
	%inner sep=0pt,%\inndist,%
	inner xsep=0.5ex,%
	inner ysep=0.2ex,%
    anchor=south east,%
    %font=\small,%
    scale=0.85,%
	text centered,%
	align=center%
]%
\tikzstyle{output}=[%
    input,%
    anchor=south west%
]%
\tikzstyle{stream}=[%
    input,%
    anchor=south,%
    pos=.5,%midway,%
    above,%
    sloped,%
]%