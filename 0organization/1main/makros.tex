%________________________________________________________________________
%------------------------------------------------------------------------
%					Part (I)
%/\/\/\/\/\/\/\/\/\/\/\/\/\/\/\/\/\/\/\/\/\/\/\/\/\/\/\/\/\/\/\/\/\/\/\/\
%
%
%
%
% own command to get a newline, without this disturbing underfull \hbox
% still beware of the "Glue" from the set \parskip height plus x minus y
% which in fact allows latex to stretch the space between paragraphs
% (by default up to 1pt)
\newcommand{\nl}{\par\noindent}% Abk. für NewLine
% and here a command to create really a new paragraph
% inserts an additional free line
\newcommand{\np}{\par\vspace{\baselineskip}}% Abk. für NewParagraph
\newcommand{\npi}{\par\vspace{\baselineskip}\noindent}% Abk. für NewParagraph_noIndent
% Zu benutzen also so:
% Neue Zeile ohne Einrückung: \nl
% Neue Zeile mit Einrückung: Freie Zeile im Editor
% Neuer Absatz mit freier Zeile dazwischen: \na
\newcommand{\greenuline}[1]{\colorlet{temp}{.}\color{green}\underline{\color{temp}#1}\color{temp}\xspace}%
%
%
%
%
%---------------------------------------------
% Vertically aligning Text of different size
%---------------------------------------------
% USAGE Example: 
%    \npi
%    {\LARGE SomeUng --\raiseC{\normalsize -- OtherdgM-- }\raiseT{\normalsize -- OtherdgM-- }\raiseT{\normalsize -- OtherdM}}
%    \npi
%    SomeUng --\raiseC{\tiny -- OtherdgM-- }\raiseT{\tiny -- OtherdgM-- }\raiseT{\tiny -- OtherdM}
%---------------------------------------------
\newsavebox{\verticalAlign}%
\newlength{\verticalAlignHeight}%
\newcommand{\textOccupyHeight}{%
    %\ht\strutbox%
    %\heightof{\strut}%
    %\heightof{AMXgpqj}%
    %\ht\vphantom{AMXgpqj}%
    \heightof{\vphantom{AMXgpqj}}%
}%
% \depthof{#1}
\newcommand*{\raiseT}[1]{%
    \setlength{\verticalAlignHeight}{\textOccupyHeight}%
    \sbox{\verticalAlign}{#1}%
    \raisebox{\verticalAlignHeight-\ht\verticalAlign}{\usebox{\verticalAlign}}%
}%
\newcommand*{\raiseC}[1]{%
    \setlength{\verticalAlignHeight}{\textOccupyHeight}%
    \sbox{\verticalAlign}{#1}%
    \raisebox{0.5\verticalAlignHeight-0.5\ht\verticalAlign}{\usebox{\verticalAlign}}%
}%
%
\newcommand*{\adjustT}[1]{%
    \adjustbox{valign=t}{#1}%
}%
\newcommand*{\adjustC}[1]{%
    \adjustbox{valign=c}{#1}%
}%
%
%
%
%
%
%
%________________________________________________________________________
%------------------------------------------------------------------------
%			Footnotes - More elaborate use
%/\/\/\/\/\/\/\/\/\/\/\/\/\/\/\/\/\/\/\/\/\/\/\/\/\/\/\/\/\/\/\/\/\/\/\/\
% Fußnote wiederholt referenzieren
\newcommand{\savefootnote}[2]{\footnote{\label{#1}#2}}%
\newcommand{\repeatfootnote}[1]{\textsuperscript{\ref{#1}}\ifMultipleFootnote}%
%
\newcommand{\savefootnoteNoMark}[2]{\footnote{\label{#1}#2}}%
% Beispiel
% %  First sentence.\savefootnote{footnote-label}{footnote content}
% %  Second sentence.\repeatfootnote{footnote-label}
%
%
% Own Makro für footnote ohne Marker
% Unmarked Footnote
\makeatletter%
%\def\blfootnote{\xdef\@thefnmark{}\@footnotetext}
\def\blfootnote{\gdef\@thefnmark{}\@footnotetext}%
\makeatother%
% Beispiel:
%  \blfootnote{No number here!}
%
% Another Variant for a Blank-Footnote
\newcommand\blankfootnote[1]{%
    \begingroup%
    \renewcommand\thefootnote{}\footnote{#1}%
    \addtocounter{footnote}{-1}%
    \endgroup%
}%
%
% This places a footnote on the bottom of the page but does not leave a Mark where the footnote is defined.
%     It inserts the number nicely into the ordinary numbering series of normal \footnote{}
% Arguments:
%  1 - A label for later referencing of this footnote, i.e. creating a mark
%  2 - The actual footnote
\newcommand{\footnoteplace}[2]{%
    \addtocounter{footnote}{1}%
    \footnotetext[\thefootnote]{\label{#1}#2}%
}%
% This creates a marker corresponding to a formerly "placed" footnote, using its declared label
\newcommand{\footnoteref}[1]{\repeatfootnote{#1}}%
%
% Auxiliary stuff. (Formats footnote-references nicely with comma, in case multiple are set in succession)
\newcommand\nextToken\relax%
\newcommand\tokenIsFootnoteRef{%
    \ifx\footnoteref\nextToken%
        \textsuperscript{,}%
    \else\ifx\repeatfootnote\nextToken%
        \textsuperscript{,}%
    \else\ifx\savefootnote\nextToken%
        \textsuperscript{,}%
    \else\ifx\footnote\nextToken%
        \textsuperscript{,}%
    \fi\fi\fi\fi%
}%
\newcommand\ifMultipleFootnote{\futurelet\nextToken\tokenIsFootnoteRef}%
%/\/\/\/\/\/\/\/\/\/\/\/\/\/\/\/\/\/\/\/\/\/\/\/\/\/\/\/\/\/\/\/\/\/\/\/\
%				Footnotes done
%------------------------------------------------------------------------
%________________________________________________________________________
%
%
%
%
%
%
% Makro für Kreis um Text
\newcommand\circlearound[1]{\tikz[baseline]\node[draw,shape=circle,anchor=base,inner sep=0.3ex]{#1};}%
\newcommand\lipsaround[1]{\tikz[baseline]\node[draw,shape=ellipse,anchor=base,inner sep=0.3ex]{#1};}%
%
\newcommand\lipsaroundCol[3]{\tikz[baseline]\node[draw=#2,fill=#3,shape=ellipse,anchor=base,inner sep=0.3ex]{#1};}%
% fill oder draw = none für farblos
%
\newcommand\rectdotaround[2]{\tikz[baseline]\node[draw=#1,thick, loosely dotted,shape=rectangle,anchor=base,inner sep=0.5ex]{#2};}%
%
%
%
%
%Some Symbols. (Using package 'amssymb', 'pifont')
\newcommand{\checksign}{\ding{51}}%
\newcommand{\crosssign}{\ding{55}}%
%
%
%
\newcommand{\HRule}{\rule{\linewidth}{0.5mm}}%
%
%
%
%
%
% Requires Packages:
%  pict2e, fp, xparse, tikz, mathtools, cals, amssymb
%=================================
% Some More Symbols
%================================================
% \newcommand{\compoundarrow}{\rule[0.5ex]{0.6ex}{0.6ex}\scriptsize\raise2pt\hbox{$\!{\blacktriangleright}$}}%
\newcommand{\compoundarrow}{\tikz{%
\def\compoundarrowDimXOne{1ex}%
\def\compoundarrowDimXTwo{1.4ex}%
\def\compoundarrowDimYOne{0.6ex}%
\def\compoundarrowDimYTwo{1.2ex}%
\path[fill,line join=miter,rounded corners=0](0,0)--++(\compoundarrowDimXOne,0)--++(0,-\compoundarrowDimYOne)--++(\compoundarrowDimXTwo,\compoundarrowDimYTwo)--++(-\compoundarrowDimXTwo,\compoundarrowDimYTwo)--++(0,-\compoundarrowDimYOne)--++(-\compoundarrowDimXOne,0)--cycle;%
}}%
\newcommand{\tikzarrow}{\tikz{\usetikzlibrary{shapes.arrows}\node[single arrow,scale=0.2,inner sep=2ex,fill]at(0,0){};}}%
%
\makeatletter%
%ToDo: A filling of the path is still missing for the \blackpointright. Currently it's identical to \pointright
\DeclareRobustCommand{\blackpointright}{%
    \mathrel{\mathpalette\blackpoint@right\relax}%
}%
\newcommand{\blackpoint@right}[2]{%
    \vcenter{\hbox{\blackpoint@@right{#1}}}%
}%
\newcommand{\blackpoint@@right}[1]{%
    \sbox\z@{$\m@th#1\blacktriangleright$}%
    \setlength{\unitlength}{\ht\z@}%
    \setlength{\dimen@}{\wd\z@}%
    \linethickness{%
        \ifx#1\displaystyle 0.09\unitlength%
        \else\ifx#1\textstyle 0.09\unitlength%
        \else\ifx#1\scriptstyle 0.11\unitlength%
        \else 0.13\unitlength\fi\fi\fi%
    }%
    \edef\blackpoint@wd{\fpeval{2*(\dimen@/\unitlength)}}%
    \begin{picture}(\blackpoint@wd,1)%
    \roundjoin%
    \polyline(0,0.5)(0,0)(\blackpoint@wd,0.5)(0,1)(0,0.2)\closepath\fillpath%
    \end{picture}%
}%
%
\DeclareRobustCommand{\pointright}{%
    \mathrel{\mathpalette\point@right\relax}%
}%
\newcommand{\point@right}[2]{%
    \vcenter{\hbox{\point@@right{#1}{\triangleright}}}%
}%
\newcommand{\point@@right}[2]{%
    \sbox\z@{$\m@th#1\triangleright$}%
    \setlength{\unitlength}{\ht\z@}%
    \setlength{\dimen@}{\wd\z@}%
    \linethickness{%
        \ifx#1\displaystyle 0.09\unitlength%
        \else\ifx#1\textstyle 0.09\unitlength%
        \else\ifx#1\scriptstyle 0.11\unitlength%
        \else 0.13\unitlength\fi\fi\fi%
    }%
    \edef\point@wd{\fpeval{2*(\dimen@/\unitlength)}}%
    \begin{picture}(\point@wd,1)%
    \roundjoin%
    \polyline(0,0.5)(0,0)(\point@wd,0.5)(0,1)(0,0.2)%
    \end{picture}%
}%
\makeatother%
%USAGE:
% $A\pointright B \triangleright C$
% $\pointright\triangleright$
% $\scriptstyle\pointright\triangleright$
% $\scriptscriptstyle\pointright\triangleright$
%
%=================================
% Some More Symbols END
%================================================
%
%
%
%=================================
% Presets for Usage as Itemization Label
%================================================
% USAGE: \begin{itemize}[\labpragA]   or   \item[\labpragAsym]
%   (Standing for: Label-Prägnanz-#
%   (The \labprag# are defined to be readily usable for a whole itemization environment (i.e. comprise also a fitting left-margin)
%   (The \labprag#sym can be used for single items as they do only define the Symbol itself
\newcommand{\labpragAsym}{\raisebox{.5\height}{\textbf{\scriptsize$\rightarrow$}}}%
\SetEnumitemKey{labpragA}{label=\labpragAsym,leftmargin=1.25em}%
\newcommand{\labpragBsym}{\raisebox{.5\height}{\textbf{$\scriptstyle\Rightarrow$}}}%
\SetEnumitemKey{labpragB}{label=\labpragBsym,leftmargin=1.3em}%
\newcommand{\labpragCsym}{\raisebox{.5\height}{\scalebox{0.7}{$\pointright$}}}%
\SetEnumitemKey{labpragC}{label=\labpragCsym,leftmargin=1.3em}%
\newcommand{\labpragDsym}{\raisebox{.4\height}{\scalebox{0.7}{$\blacktriangleright$}}}%
\SetEnumitemKey{labpragD}{label=\labpragDsym,leftmargin=1.2em}%
\newcommand{\labpragEsym}{\raisebox{.1\height}{\scalebox{0.6}{\compoundarrow}}}%
\SetEnumitemKey{labpragE}{label=\labpragEsym,leftmargin=1.3em}%
%=================================
% Label-Presets END
%================================================
%
%
%
%
%
%
% Nice for use with e.g. Tables / Legends.
%     Look into the Folder 'organization/templates' for a Table Example inclusive Legend
\definecolor{geilesrot}{rgb}{ .753,  0,  0}%
\definecolor{angenehmesorange}{rgb}{ 1,  .753,  0}%
\definecolor{superduftesgruen}{rgb}{ 0,  .69,  .314}%
\definecolor{woDieSonneNieHinscheintSchwarz}{rgb}{ 0,  0,  0}%
%----------------------------------------------------------------
	\newcommand{\legcrossstraight}{\textcolor{geilesrot}{$\times$}}%
\newcommand{\legcross}{\textbf{\huge\legcrossstraight}}%
	\newcommand{\legbulletstraight}{\textcolor{angenehmesorange}{$\bullet$}}%
\newcommand{\legbullet}{\textbf{\huge\legbulletstraight}}%
	\newcommand{\legcheckstraight}{\textcolor{superduftesgruen}{$\checkmark$}}%
\newcommand{\legcheck}{\textbf{\huge\legcheckstraight}}%
%
%
%
%
%
%
%
%
%
%
%________________________________________________________________________
%------------------------------------------------------------------------
%			Verschiedene Definitionen wie Schriftzeichen
%/\/\/\/\/\/\/\/\/\/\/\/\/\/\/\/\/\/\/\/\/\/\/\/\/\/\/\/\/\/\/\/\/\/\/\/\
\newcommand{\mytildeA}{\mbox{\raise.17ex\hbox{$\scriptstyle\mathtt{\sim}$}}\relax}%
\newcommand{\mytildeB}{\textasciitilde}%
\newcommand{\mytildeC}{$\sim$}%
\newcommand{\mytildeD}{$\mathord{\sim}$}%
\newcommand{\mytildeE}{\~{}}%
\newcommand{\mytildeF}{\texttildelow}%
\newcommand{\mytildeG}{\raisebox{-.8ex}{\textasciitilde}}%
\newcommand{\mytilde}{\mytildeA}%
\newcommand{\myequalhat}{\mathrel{\widehat{=}}}%
\newcommand{\myequalhatA}{\mathrel{\stackon[1.5pt]{=}{\stretchto{\scalerel*[\widthof{=}]{\wedge}{\rule{1ex}{3ex}}}{0.5ex}}}}% Would require \usepackage{scalerel,stackengine,amsmath}
%
% Creates an ordinary text-dash ('-'), but with leaving hyphenation active for the joined words
%  (For this, 'ngerman' has to be defined (possibly unproblematically among others) by babel.sty)
\newcommand{\hyphdash}{\foreignlanguage{ngerman}{"=}}% Variant for Babel
%
% For Guillemets/Chevron to use with every Font (Some Fonts don't define a sign for them)
\newcommand{\DKguillemetL}{\xspace{\fontfamily{ptm}\selectfont »}}%
\newcommand{\DKguillemetR}{{\fontfamily{ptm}\selectfont «}\xspace}%
%/\/\/\/\/\/\/\/\/\/\/\/\/\/\/\/\/\/\/\/\/\/\/\/\/\/\/\/\/\/\/\/\/\/\/\/\
%							Definitionen fertig
%------------------------------------------------------------------------
%________________________________________________________________________
%
%
%
%
%
%
%
%
%
%
%________________________________________________________________________
%------------------------------------------------------------------------
%					Part (II)
%/\/\/\/\/\/\/\/\/\/\/\/\/\/\/\/\/\/\/\/\/\/\/\/\/\/\/\/\/\/\/\/\/\/\/\/\
%
%
%
%
% Initialisierungen wie Counter, Theorem, Variablen, Konstanten..
%
\newcommand{\stdleftbarwidth}{0.15em}%
%
\theoremstyle{break}%
\theoremheaderfont{\bfseries}%
\theorembodyfont{\normalfont}%
\theoremseparator{}%
\theorempreskip{0ex}%
\theorempostskip{0ex}%
\theoremindent0.5em%
%
%That is not quite working well, hence, instead an own solution below
%    \newtheorem{InciRem}{\protect\InciRemname}[section]%
%    \newcommand{\InciRemname}{}%Initialization
%    \addto\captionsenglish{%
%      \renewcommand{\InciRemname}{Incidential Remark}%
%    }%
%    \addto\captionsngerman{%
%      \renewcommand{\InciRemname}{Einschub}%
%    }%
\ifdef{\DenKrLayoutLanguage}{%
    \expandafter\ifstrequal\expandafter{\DenKrLayoutLanguage}{english}{%
        \newtheorem{InciRem}{Incidential Remark}[section]%
    }{%
    \expandafter\ifstrequal\expandafter{\DenKrLayoutLanguage}{ngerman}{%
        \newtheorem{InciRem}{Einschub}[section]%
    }{%
    \expandafter\ifstrequal\expandafter{\DenKrLayoutLanguage}{german}{%
        \newtheorem{InciRem}{Einschub}[section]%
    }{%
        \newtheorem{InciRem}{Incidential Remark}[section]%
    }%
    }%
    }%
}{%
    \newtheorem{InciRem}{Incidential Remark}[section]%
}%
%
%
%
%
%
%________________________________________________________________________
%------------------------------------------------------------------------
%     A manual Counter. Just to have an arbitrary counter at hand
%/\/\/\/\/\/\/\/\/\/\/\/\/\/\/\/\/\/\/\/\/\/\/\/\/\/\/\/\/\/\/\/\/\/\/\/\
\newcounter{manualCounter}%
\setcounter{manualCounter}{0}%
\newcommand{\manCountReset}{\setcounter{manualCounter}{0}}%
\DeclareDocumentCommand{\manCount}{o}{% Only 1 possible Argument (optional), which is a label for referencing
    \refstepcounter{manualCounter}%
    \IfNoValueTF{#1}{}{%
        \label{#1}%
    }%
    \arabic{manualCounter}%
}%
%------------------------------------------------------------------------
%________________________________________________________________________
%
%
%
%
%
%________________________________________________________________________
%------------------------------------------------------------------------
%			Typesetting
%/\/\/\/\/\/\/\/\/\/\/\/\/\/\/\/\/\/\/\/\/\/\/\/\/\/\/\/\/\/\/\/\/\/\/\/\
% Usage: \DenKrSubfloat{<Label>}{<SubCaption>}{<FloatCode>}
%   Yes, I declared all three Arguments as mandatory, Deal with it.
\makeatletter\AddToHook{begindocument/before}{%
\@ifpackageloaded{subfig}{%
    \DeclareDocumentCommand{\DenKrSubfloat}{m m m}{%
        \subfloat[#2]{\label{#1}#3}%
    }%
}{\@ifpackageloaded{subcaption}{
    \DeclareDocumentCommand{\DenKrSubfloat}{m m m}{%
        \sbox{\tmpbox}{#3}%
        % \settowidth{\tmpwidth}{#3}%
        \settowidth{\tmpwidth}{\usebox{\tmpbox}}%
        \begin{minipage}{\tmpwidth}%
            \centering%
            \usebox{\tmpbox}%
            \protect\subcaption{#2}%
            \label{#1}%
        \end{minipage}%
    }%
}{\gerguiquote{Neither \textbf{subcaption} nor \textbf{subfig} loaded}}%
}%
}\makeatother%
%/\/\/\/\/\/\/\/\/\/\/\/\/\/\/\/\/\/\/\/\/\/\/\/\/\/\/\/\/\/\/\/\/\/\/\/\
%				Basic Typesetting END
%------------------------------------------------------------------------
%________________________________________________________________________
%
%
%
%
%________________________________________________________________________
%------------------------------------------------------------------------
%			Sonder-Formatierungen (Randnotiz, Trenn-Markierung)
%/\/\/\/\/\/\/\/\/\/\/\/\/\/\/\/\/\/\/\/\/\/\/\/\/\/\/\/\/\/\/\/\/\/\/\/\
%
% Own redefinition of leftbar environment
% Einfach ums schöner zu machen
\newenvironment{myleftbar}[2]%[1=0.5pt,2=5pt]%
{%
    % \setlength{\tempheight}{\topsep}%
    % \setlength{\topsep}{0ex}%
    \vspace*{-0.95\topsep}%
    \def\FrameCommand{\hspace{#2} \vrule width #1 \hspace{0.0em}}%
    \MakeFramed{\advance\hsize-\width\FrameRestore}%
}%
{%
    \endMakeFramed%
    \vspace*{-\topsep}%
    % \setlength{\topsep}{\tempheight}% Doesn't work
}%
%
%
%
%
%
\DeclareDocumentCommand{\sidenote}{%
m O{Sidenote} O{0pt}%
}{%
\vspace{#3}%
\begin{myleftbar}{\stdleftbarwidth}{0.5em}%
\begin{InciRem}[#2]%
#1%
\end{InciRem}%
\end{myleftbar}%
\vspace{#3}%
}%
% USAGE: \sidenote{<Text-Inhalt>}[<Label>][<Whitespace-Before-and-After>]
%    e.g.: \sidenote{Eine Beispiel Notiz}%[Just some (optional) Example][0.5\baselineskip]
%
%
% sidenote Same-Page
\DeclareDocumentCommand{\sidenoteSP}{%
m O{Sidenote} O{0pt}%
}{%
\vspace{#3}%
\begin{samepage}%
\begin{myleftbar}{\stdleftbarwidth}{0.25em}%
\begin{InciRem}[#2]%
#1%
\end{InciRem}%
\end{myleftbar}%
\end{samepage}%
\vspace{#3}%
}%
%
%
%
\newcommand{\trennstern}{%
\vspace{\baselineskip}\hrule%
{\noindent\hspace{2em}*\hfill*\hfill*\hspace{2em}}%
\hrule\vspace{\baselineskip}%
}%
%
%/\/\/\/\/\/\/\/\/\/\/\/\/\/\/\/\/\/\/\/\/\/\/\/\/\/\/\/\/\/\/\/\/\/\/\/\
%							Sonder-Formatierungen END
%------------------------------------------------------------------------
%________________________________________________________________________






% Makro to define a chapter without numbering, but with Entry in Table of Contents
% Chapter without numbering to TOC
% Arguments:
% #1 - Optional. Alternative Name for TOC
% #2 - Mandatory. Name of the Chap/Sec/Subsec itself
% #3 - Optional. Label
%
%					%				Old Version					%
% 					% 					% \newcommand{\chapnt}[2]{%
% 					% 					% 	\chapter*{#1}%
% 					% 					% 	\label{#2}%
% 					% 					% 	\addcontentsline{toc}{chapter}{\nameref{#2}}%
% 					% 					% 	\markboth{\nameref{#2}}{\nameref{#2}}%
% 					% 					% 	\addtocounter{chapter}{1}%
% 					% 					% }%
\DeclareDocumentCommand{\chapnt}{%
O{#2} m o%
}{%
	\chapter*{#2}%
	\IfValueTF{#3}{%
		\label{#3}%
	}{}%
% 	\addcontentsline{toc}{chapter}{\nameref{#3}}%
% 	\markboth{\nameref{#2}}{\nameref{#3}}%
	\addcontentsline{toc}{chapter}{#1}%
	\markboth{#1}{#1}%
	\addtocounter{chapter}{1}%
}%
% Usage: \chapnt{Chapter Name}{Label}
%
% And for Section
%	Sets the \rightmark. Remember to change the Brackets for \leftmark if needed
%	With empty Bracket it sets it blank
%					%				Old Version					%
% 					% 					% \newcommand{\secnt}[2]{%
% 					% 					% 	\section*{#1}%
% 					% 					% 	\label{#2}%
% 					% 					% 	\addcontentsline{toc}{section}{\nameref{#2}}%
% 					% 					% 	\markboth{}{\nameref{#2}}%
% 					% 					% }%
\DeclareDocumentCommand{\secnt}{%
O{#2} m o%
}{%
	\section*{#2}%
	\IfValueTF{#3}{%
		\label{#3}%
	}{}%
	\addcontentsline{toc}{section}{#1}%
% 	\markboth{\leftmark}{#1}%
	\markright{#1}%
}%
%
% And for SubSection
%					%				Old Version					%
% 					% 					% \newcommand{\subsecnt}[2]{%
% 					% 					% 	\subsection*{#1}%
% 					% 					% 	\label{#2}%
% 					% 					% 	%\addcontentsline{toc}{subsection}{\protect\numberline{}#1}%This would add a whitespace, where normally the section-numbering would be.
% 					% 					% 	\addcontentsline{toc}{subsection}{#1}%
% 					% 					% 	\markboth{}{\nameref{#2}}%
% 					% 					% }%
\DeclareDocumentCommand{\subsecnt}{%
O{#2} m o%
}{%
	\subsection*{#1}%
	\IfValueTF{#3}{%
		\label{#3}%
	}{}%
	%\addcontentsline{toc}{subsection}{\protect\numberline{}#1}%This would add a whitespace, where normally the section-numbering would be.
	\addcontentsline{toc}{subsection}{#1}%
}%
%
% Chapter with Subtitle
\newcommand\chaptersubt[2]{\chapter%
  [#1\hfil\hbox{}\protect\linebreak{\itshape#2}]%
  {#1\\[2ex]\Large\itshape#2}%
}%
\newcommand\sectionsubt[2]{\section%
  [#1\hfil\hbox{}\protect\linebreak{\itshape#2}]%
  {#1\\[2ex]\Large\itshape#2}%
}%
%
%
%
%
%
%
%
%
%
%WorkingStateDivider
\newcommand{\wdiv}{\vspace{0.5\baselineskip}\npi\hspace*{\fill}====================================================\hspace*{\fill}\nl\#\#\#\#\#\#\#\#\#\#\#\#\#\#\#\#\#\#\#\#\#\#\#\#\hfill\#\#\#\#\#\#\#\#\#\#\#\#\#\#\#\#\#\#\#\#\#\#\#\#\nl}%
%
%
%
% Eventuell Labelprüfung vor \cref@gettype einbauen:
%	(ifcsdef aus der etoolbox)
% Macros for uniform Formatting of References to Document-Elements (Figure, Equation, Algorithm, Table)
\newcommand{\abbref}[1]{\hyperref[#1]{Abbildung~\ref*{#1}}}%
\newcommand{\abbrefzwei}[2]{Abbildungen~\ref{#1} \& \ref{#2}}%
\newcommand{\figref}[1]{\hyperref[#1]{Figure~\ref*{#1}}}%
\newcommand{\figreftwo}[2]{Figures~\ref{#1} \& \ref{#2}}%
\newcommand{\absref}[1]{\hyperref[#1]{Abschnitt~\ref*{#1}}}%
\newcommand{\secref}[1]{\hyperref[#1]{Section~\ref*{#1}}}%
\newcommand{\tablref}[1]{\hyperref[#1]{Tabelle~\ref*{#1}}}%
\newcommand{\tabref}[1]{\hyperref[#1]{Table~\ref*{#1}}}%
\newcommand{\algorefGer}[1]{\hyperref[#1]{Algorithmus~\ref*{#1}}}%
\newcommand{\algoref}[1]{\hyperref[#1]{Algorithm~\ref*{#1}}}%
\newcommand{\gleref}[1]{\hyperref[#1]{Gleichung~\ref*{#1}}}%
\newcommand{\equref}[1]{\hyperref[#1]{Equation~\ref*{#1}}}%
% - - - - - - - - - - - - - - - -
% To do it *automatically*, using "Cleverref"
\newcommand{\entityref}[1]{\hyperref[#1]{\cref*{#1}}}%
\newcommand{\eleref}[1]{\entityref{#1}}%
\newcommand{\typeref}[1]{\entityref{#1}}%
% - - - - - - - - - - - - - - - -
% Including the *Name/Title/Caption* of the Element
\makeatletter%
\newcommand{\numnamref}[1]{%
    %\cref@gettype{#1}{\currlabtype}%
    \hyperref[#1]{\cref*{#1}} (\nameref{#1})%
}%
\makeatother%
\newcommand{\entitynumnamref}[1]{\numnamref{#1}}%
\newcommand{\elenumnamref}[1]{\numnamref{#1}}%
\newcommand{\typenumnamref}[1]{\numnamref{#1}}%
% - - - - - - - - - - - - - - - -
% Nameref with some fancy formatting
\newcommand{\formattednamref}[1]{{%
    \color{DenKrColor_Highlight_Emerald}\nameref{#1}%
}}%
\newcommand{\entityformattednamref}[1]{\formattednamref{#1}}%
\newcommand{\eleformattednamref}[1]{\formattednamref{#1}}%
\newcommand{\typeformattednamref}[1]{\formattednamref{#1}}%
% - - - - - - - - - - - - - - - -
% The formatted above + quotation
\newcommand{\formatquotnamref}[1]{%
    \gerguiquotel{\formattednamref{#1}}%
}%
%
% - - - - - - - - - - - - - - - -
% To create a Reference with Custom-Text
% 1: The Hyperlink-Destination, i.e. just "sec:target" not "\pageref{sec:target}
% 2: The text that shall be displayed.
% -> The entire text is clickable and leads to the link's destination
\newcommand{\refCustom}[2]{%
    \hyperref[#1]{#2}%
}%
%
% - - - - - - - - - - - - - - - -
% - Legacy stuff
% - - (Better just ignore. Only still here, because I am not able to delete old code. '^.^  I might have a problem there...)
%--------------------------------
\newcommand{\abnamref}[1]{Abbildung~\ref{#1} (\nameref{#1})}%
\newcommand{\fignamref}[1]{Figure~\ref{#1} (\nameref{#1})}%
% \newcommand{\absref}[1]{\textbf{Abschnitt~\ref{#1}}}%
% Und das gleiche inklusive \nameref
\makeatletter%
\newcommand{\absamref}[1]{%
 	\cref@gettype{#1}{\currlabtype}%
	\IfSubStr{\currlabtype}{part}{%
		\textsf{\cref{#1} - \nameref{#1}}%
	}{%
	\IfSubStr{#1}{subsubsec}{%
%	'subsubsection' isn't functional with gettype.
%	This only goes deep until subsection
%	So be careful and append a preceding 'subsubsec'
%	on the label
		Unterabschnitt >>\nameref{#1}<< aus \ref{#1}%
	}{%
		\cref{#1} (\nameref{#1})%
	}%
	}%
}%
\makeatother%
\makeatletter%
\newcommand{\secnamref}[1]{%
 	\cref@gettype{#1}{\currlabtype}%
	\IfSubStr{\currlabtype}{part}{%
		\textsf{\cref{#1} - \nameref{#1}}%
	}{%
	\IfSubStr{#1}{subsubsec}{%
%	'subsubsection' isn't functional with gettype.
%	This only goes deep until subsection
%	So be careful and append a preceding 'subsubsec'
%	on the label
		Section >>\nameref{#1}<< of \ref{#1}%
	}{%
		\cref{#1} (\nameref{#1})%
	}%
	}%
}%
\makeatother%
% \newcommand{\absamref}[1]{\textbf{Abschnitt~\ref{#1}} (\nameref{#1})}%
% Für 2 Abschnitte zugleich
\makeatletter%
\newcommand{\absrefzwei}[2]{%
	\cref@gettype{#1}{\currlabtype}%
	\IfSubStr{\currlabtype}{chapter}{%
		Kapiteln~\ref{#1} \& \ref{#2}%
	}{%
		Abschnitten~\ref{#1} \& \ref{#2}%
	}%
}%
\makeatother%
\makeatletter%
\newcommand{\absamrefzwei}[2]{%
	\cref@gettype{#1}{\currlabtype}%
	\IfSubStr{\currlabtype}{chapter}{%
		Kapiteln~\ref{#1} (\nameref{#1}) \&%
			\ref{#2} (\nameref{#2})%
	}{%
		Abschnitten~\ref{#1} (\nameref{#1}) \&%
			\ref{#2} (\nameref{#2})%
	}%
}%
\makeatother%
% Absatz Name Reference Zwei Short -- means without the subsequent 'n' -
% i.e. Kapitel instead of Kapiteln
\makeatletter%
\newcommand{\absamrefzweis}[2]{%
	\cref@gettype{#1}{\currlabtype}%
	\IfSubStr{\currlabtype}{chapter}{%
		Kapitel~\ref{#1} (\nameref{#1}) \&%
			\ref{#2} (\nameref{#2})%
	}{%
		Abschnitte~\ref{#1} (\nameref{#1}) \&%
			\ref{#2} (\nameref{#2})%
	}%
}%
\makeatother%
% \newcommand{\absrefzwei}[2]{\textbf{Abschnitten~\ref{#1}} \& \ref{\ref{#2}}}%
% \newcommand{\absamrefzwei}[2]{\textbf{Abschnitten~\ref{#1}} (\nameref{#1}) \& \textbf{\ref{#2}} (\nameref{#2})}%
%
%
%
%
%
%=================================
%  Section Heading & Glossaries
%   -> You can't just use a \gls{} Command in the Heading of a Chapter, Section, etc.
%      This would cause a Warning about "Token not allowed [...]" (because of adding hyperlink (an unsupported special character) to the bookmarks)
%      Instead using \texorpdfstring{}{} specifies different strings to be placed in the actual document or as bookmark
%      Combining this with alterantive \gls cmds like \glsentrytext{} instead of \glstext{} works around this problem
%   -> So, for making this process easier: Use the following Macro as opposed hacking something yourself.
%   -> Or, if you prefer, right below this Comment-Block, you have a "Boilerplate". Use this instead of a simple \gls[...]{} (adjust 'text', 'desc', 'plural', etc. as required)
%================================================
% \texorpdfstring{\glstext{#2}}{\glsentrytext{#2}}
\makeatletter%
\newcommand{\glstextHead}[1]{%
    \expandafter\texorpdfstring{\expandafter\glsentrytext{#1}}{\expandafter\glsentrytext{#1}}%
}%
\newcommand{\glsdescHead}[1]{%
    \expandafter\texorpdfstring{\expandafter\glsentrydesc{#1}}{\expandafter\glsentrydesc{#1}}%
}%
\newcommand{\glstextplHead}[1]{%
    \expandafter\texorpdfstring{\expandafter\glsentryplural{#1}}{\expandafter\glsentryplural{#1}}%
}%
\newcommand{\glsdescplHead}[1]{%
    \expandafter\texorpdfstring{\expandafter\glsentrydescplural{#1}}{\expandafter\glsentrydescplural{#1}}%
}%
\makeatother%
%
%=================================
%  More Glossaries-Cmds
%================================================
% Printing an "Indefinite Article" (Unbestimmter Artikel)
%    For printing an Article before a \gls{} matching is expansion (c.f.: "a Software Definition"  vs.  "an SD")
\newcommand{\glsInArticle}[1]{%
    \ifglsused{#1}{an}{a}%
}%
%
%
%
%
%
%
%
%=================================
% The romaji Spoiler-Environment
%      requires package 'ocgx' / 'ocgx2'
%================================================
%	Button at Start of Env
\newcounter{romaji}%
\newenvironment{romaji}%
{\stepcounter{romaji}%
\begin{minipage}{\textwidth-1ex}%
\switchocg{romaji\arabic{romaji}}{\fbox{\tiny ローマ字}}\begin{ocg}{Romaji \arabic{romaji}}{romaji\arabic{romaji}}{0}%
\footnotesize}%
%---------------------------------------------
{\par\end{ocg}%
\end{minipage}}%
%================================================
%	Button at End of Env
% 		\newcounter{romaji}%
% 		\newenvironment{romaji}%
% 		{\stepcounter{romaji}%
% 		\begin{minipage}{\textwidth-1ex}%
% 		\begin{ocg}{Romaji \arabic{romaji}}{romaji\arabic{romaji}}{0}%
% 		\footnotesize}%
% 		%---------------------------------------------
% 		{\end{ocg}\switchocg{romaji\arabic{romaji}}{\fbox{\tiny ローマ字}}%
% 		\end{minipage}}%
%================================================
\newcommand{\ropre}{\par\vspace{0.2\baselineskip}\noindent}% Romaji-Pre
%
%
%
%
%
%
%
%
%
%
%
%
%
%
%________________________________________________________________________
%------------------------------------------------------------------------
%			Japanische Schrift
%/\/\/\/\/\/\/\/\/\/\/\/\/\/\/\/\/\/\/\/\/\/\/\/\/\/\/\/\/\/\/\/\/\/\/\/\
\makeatletter%
\newcommand{\jap}[1]{\begin{CJK}{UTF8}{min}#1\end{CJK}}%
\makeatother%
%/\/\/\/\/\/\/\/\/\/\/\/\/\/\/\/\/\/\/\/\/\/\/\/\/\/\/\/\/\/\/\/\/\/\/\/\
%							Japanisch fertig
%------------------------------------------------------------------------
%________________________________________________________________________
%
%
%
%
%
%
%
%
%
%
%
%
%
%
% Optionen für fontseries und fontshape:
% Series, any combination of weight and width is [in theory] possible:
% weight                    width
% Ultra Light       ul      Ultra Condensed     uc    
% Extra Light       el      Extra Condensed     ec    
% Light             l       Condensed            c      
% Semi Light        sl      Semi Condensed      sc    
% Medium (normal)   m
% Semi Bold         sb      Semi Expanded       sx    
% Bold              b       Expanded             x 
% Extra Bold        eb      Extra Expanded      ex 
% Ultra Bold        ub      Ultra Expanded      ux
% %
% Shape:
% upright (normal)   n 
% italic             it
% slanted/oblique    sl 
% small caps         sc
% upright italic     ui
% outline            ol 
%
%Inline English
%Zum anders formatieren von nicht übersetzten, englischen Begriffen
\newcommand*{\ien}[1]{%
	#1%
% 	{\small\fontfamily{phv}\fontseries{m}\fontshape{n}\selectfont#1}%
}%
%
%Inline Code
%Zum eigens formatieren von kleinen Code-Fetzen
\newcommand*{\ico}[1]{%
	\texttt{#1}%
}%
%
%
%
%
% Kommando-Zeilen Syntax
% The one with the 'b' suffix is for brackets only. It does not format anything,
% but add the brackets. Its something like for "inner-use". You can use it in
% wrapped makros for example.
\newcommand*{\regexcmd}[2]{% Regular Expression Command Line
	\IfSubStr{#1}{id}{%
		\texttthyph{\textcolor{ColorRegExCmdID}{<#2>}}%
	}{\IfSubStr{#1}{auswahl}{%
		\texttthyph{\textcolor{ColorRegExCmdSelect}{[#2]}}%
	}{\IfSubStr{#1}{mehrfach}{%
		\texttthyph{\textcolor{ColorRegExCmdMultiSelect}{\{#2\}}}%
	}{\IfSubStr{#1}{option}{%
		\texttthyph{-\textcolor{ColorRegExCmdOption}{$\|$#2$\|$}}%
	}{%
	}}}}%
}%
\newcommand*{\regexcmdb}[2]{% Regular Expression Command Line
	\IfSubStr{#1}{id}{%
		<#2>%
	}{\IfSubStr{#1}{auswahl}{%
		[#2]%
	}{\IfSubStr{#1}{mehrfach}{%
		\{#2\}%
	}{\IfSubStr{#1}{option}{%
		-$\|$#2$\|$%
	}{%
	}}}}%
}%
%
%
%
%
%
\definecolor{marginparcolor}{RGB}{31,73,125}%
% Neuer MarginPar, für Schriftgröße
\newcommand*{\Marginpar}{}% Anweisung "reservieren"
% Das "Standard-Kommando" in neuem Kommando mit großem 'M' retten:
\let\Marginpar\marginpar% \MarginPar ist jetzt dasselbe wie \marginpar
\renewcommand*{\marginpar}[2][]{{% ein optionales und ein normales Argument
\expandafter\renewcommand*{\glstextformat}[1]{\color{marginparcolor}##1}%
  \ifstr{#1}{}{%
  	\Marginpar{\footnotesize\raggedright \textcolor{marginparcolor}{#2}}%
  }{%
    \Marginpar[{\footnotesize\raggedright \textcolor{marginparcolor}{#1}}]%
    	{\footnotesize\raggedright \textcolor{marginparcolor}{#2}}%
  }%
}}% Double Braces to create a scope for the redefinition of glstextformat
%
%
%
%
%
%________________________________________________________________________
%------------------------------------------------------------------------
%							Makros für Cleveref
%/\/\/\/\/\/\/\/\/\/\/\/\/\/\/\/\/\/\/\/\/\/\/\/\/\/\/\/\/\/\/\/\/\/\/\/\
%Get Type of \ref with help of cleveref
\def\currlabtype{}% Variable reservieren, um im ChapRef Makro den Typ zu speichern
%/\/\/\/\/\/\/\/\/\/\/\/\/\/\/\/\/\/\/\/\/\/\/\/\/\/\/\/\/\/\/\/\/\/\/\/\
%							Cleveref fertig
%------------------------------------------------------------------------
%________________________________________________________________________
%
%
% 
%
%
%
%
% INFO:
% To print a value (length, printlength)
% \the\value
% e.g.: \the\textwidth
% e.g.: \the\linewidth
% e.g.: \the\columnwidth
%
%
%
%
%
%
%________________________________________________________________________
%------------------------------------------------------------------------
%			Tikz Macros
%/\/\/\/\/\/\/\/\/\/\/\/\/\/\/\/\/\/\/\/\/\/\/\/\/\/\/\/\/\/\/\/\/\/\/\/\
%
\newcommand{\tikzpicturescale}{1.0}%
\newcommand{\tikzpicturescaleboxfactor}{1.0}%
%
% Variable für die Breite/Höhe von eingebundenen Bildern
% Verwendet im folgenden Makro
\newlength{\tikzwidth}%
\newlength{\tikzheight}%
\newlength{\textheightscaled}%
\newlength{\textwidthactual}%
\newcommand{\tikzHeightFactorArgument}{1}%
\newcommand{\tikzWidthFactorArgument}{1}%
\newlength{\tikzAimedHeight}%
\newlength{\tikzAimedWidth}%
\newlength{\captionHeight}%
\newcommand{\tikzFloatPositioning}{!ht}%
% \textheightscaled=0.96\textheight%
% Makro für die Tikz Abbildungen
% Arg#1 (Optional): Includestandalone-Mode
% Arg#2: Pic-Name (Location: \tikzFilesPath//NAME.tex) % This Macro can be defined somewhere, by you. In my Template it is found inside the Preamble right by the inclusion of the tikz-Package, together with \includestandalonedefaultmode
% Arg#3 (Optional): Alternative Directory to take the picture from. This macro "actually" only takes the Picture-Name and tries to take it from the directory "\tikzFilesPath" aka set to "\DenKrTikzRootDir". Passing a value as this optional Argument, overrides the directory. (Pass without trailing '\')
% Arg#4: Bildbeschriftung
% Arg#5 (Optional): Alternatives Caption (TableOfContents usw.)
% Arg#6: Number of Lines des Caption
% Arg#7 (Optional): Liste mit folgendem Inhalt. Parameter zu Skalierung und Positionierung
%			#1: Faktor der Seitenhöhe, der nicht überschritten werden soll 
%			#2: Faktor der Seitenbreite, der nicht überschritten werden soll
%			#3: Positionierungsmodus. Bsp.: !ht, !htpb, H
%	  Beispiel für ein Listen-Format: [1,1,!htp]
% Arg#8: Label für referencing
% Arg#9 (Optional): Rotations-Winkel für das Bild
%% Möglichkeiten für den includestandalone-mode: tex, buildnew
%% The Default Macro: See Definition in Header (should be buildnew)
%% Use the optinal Argument with "tex", to make the tikz-Pics inline and hence
%% compiled every Runtime, without pre-rendered picture. buildnew creates
%% pre-rendered Pictures and includes them like standard-png; only rendered, if
%% tex-file is newer than the picture
\DeclareDocumentCommand{\tikzabb}{%
O{\includestandalonedefaultmode} m O{\tikzFilesPath} m o m >{\SplitList{,}}O{1,1,!ht} m O{0}%
}{%
\def\efigure{\begin{figure}}%
\def\efigureend{\end{figure}}%
\SplitListScalePos#7\relax%
%\textheightscaled=\textheight-\parskip-\abovecaptionskip-\belowcaptionskip-\baselineskip%
\textheightscaled=\dimexpr\textheight-#6\baselineskip-\parskip-\abovecaptionskip-\belowcaptionskip\relax%
\textwidthactual=\dimexpr\linewidth\relax%
\ifthenelse{#9=90}{%
	\tikzAimedHeight=\tikzWidthFactorArgument\textwidthactual%
	\tikzAimedWidth=\tikzHeightFactorArgument\textheightscaled%
}{%
	\ifthenelse{#9=-90}{%
		\tikzAimedHeight=\tikzWidthFactorArgument\textwidthactual%
		\tikzAimedWidth=\tikzHeightFactorArgument\textheightscaled%
	}{%
		\tikzAimedHeight=\tikzHeightFactorArgument\textheightscaled%
		\tikzAimedWidth=\tikzWidthFactorArgument\textwidthactual%
	}%
}%
\settowidth{\tikzwidth}{\includestandalone{#3/#2}}%
\settoheight{\tikzheight}{\includestandalone{#3/#2}}%
\tikzwidth=\tikzpicturescaleboxfactor\tikzwidth%
\tikzheight=\tikzpicturescaleboxfactor\tikzheight%
\ifthenelse{\tikzAimedHeight<\tikzheight}{%
	\tikzheight=\tikzAimedHeight%
	\settowidth{\tikzwidth}{\includestandalone[height=\tikzheight]{#3/#2}}%
	\ifthenelse{\tikzAimedWidth<\tikzwidth}{%
		\tikzwidth=\tikzAimedWidth%
		\expandafter\efigure\expandafter[\tikzFloatPositioning]%
			\centering%
% 	 		\fbox{%
			\includestandalone[width=\tikzwidth,angle=#9,mode=#1]{#3/#2}%
% 	 		}%
			\IfValueTF{#5}{%
				\caption[#5]{#4}%
			}{%
				\caption{#4}%
			}%
			\label{#8}%
		\efigureend%
	}{%
		\expandafter\efigure\expandafter[\tikzFloatPositioning]%
			\centering%
% 	 		\fbox{%
			\includestandalone[height=\tikzheight,angle=#9,mode=#1]{#3/#2}%
% 	 		}%
			\IfValueTF{#5}{%
				\caption[#5]{#4}%
			}{%
				\caption{#4}%
			}%
			\label{#8}%
		\efigureend%
	}%
}{%
	\ifthenelse{\tikzAimedWidth<\tikzwidth}{\tikzwidth=\tikzAimedWidth}{}%
	\expandafter\efigure\expandafter[\tikzFloatPositioning]%
		\centering%
% 		\fbox{%
		\includestandalone[width=\tikzwidth,mode=#1]{#3/#2}%
% 		}%
		\IfValueTF{#5}{%
			\caption[#5]{#4}%
		}{%
			\caption{#4}%
		}%
		\label{#8}%
	\efigureend%
}%
}%
%
\newcommand{\SplitListScalePos}[3]{%
	\renewcommand{\tikzHeightFactorArgument}{#1}%
	\renewcommand{\tikzWidthFactorArgument}{#2}%
	\renewcommand{\tikzFloatPositioning}{#3}%
}%
%
% Nice Makro, which allows to check if a node with certain name exists.
% Gives you an If-Then-Else Operator for use like this:
% \ifnodedefined{Node-name without brackets}{Wenn existent, dann das}{else}
\makeatletter%
\long\def\ifnodedefined#1#2#3{%
  \@ifundefined{pgf@sh@ns@#1}{#3}{#2}}%
%
\newcommand\aeundefinenode[1]{%%
  \expandafter\ifx\csname pgf@sh@ns@#1\endcsname\relax%
  \else%
    \typeout{===>Undefining node "#1"}%
    \global\expandafter\let\csname pgf@sh@ns@#1\endcsname\relax%
  \fi%
}%
%Another to undefine nodes, that they are gone in successive pics
\newcommand\aeundefinethesenodes[1]{%
  \foreach \myn  in {#1}%
    {%
      \expandafter\aeundefinenode\expandafter{\myn}%
    }%
}%
\makeatother%
%
%________________________________________________________________________
%------------------------------------------------------------------------
%			Tikz END
%/\/\/\/\/\/\/\/\/\/\/\/\/\/\/\/\/\/\/\/\/\/\/\/\/\/\/\/\/\/\/\/\/\/\/\/\
%
%
%
%
%
%
%
\newenvironment{scope}{}{}%
%
%
%
%
%
%
%
%==========================================
%##########################################
%    Schriftgrößen
%==========================================
% \tiny
% \scriptsize
% \footnotesize
% \small
% \normalsize
% \large
% \Large
% \LARGE
% \huge
% \Huge
%##########################################
%==========================================