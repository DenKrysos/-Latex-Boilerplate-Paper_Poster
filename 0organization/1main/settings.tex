%
%
% Setups
%
%
%
%
%		New Lenghts & Variables
\newlength\textheighttemp%
% \setlength{\textheighttemp}{\textheight}%
\newlength\textwidthtemp%
% \setlength{\textwidthtemp}{\textwidth}%
\newlength\textheightstd%
\setlength{\textheightstd}{\textheight}%
\newlength\textwidthstd%
\setlength{\textwidthstd}{\textwidth}%
\newlength\textheightold%
\newlength\textwidthold%
%
\newlength\tempheight%
\newlength\tempwidth%
%
%
%
%
%
%
%
%
%
%________________________________________________________________________
%------------------------------------------------------------------------
%					Spacing around Floats
%/\/\/\/\/\/\/\/\/\/\/\/\/\/\/\/\/\/\/\/\/\/\/\/\/\/\/\/\/\/\/\/\/\/\/\/\
% You can modify the following lengths, which affect all floats.;
%    \floatsep: space left between floats (12.0pt plus 2.0pt minus 2.0pt).
%    \textfloatsep: space between last top float or first bottom float and the text (20.0pt plus 2.0pt minus 4.0pt).
%    \intextsep : space left on top and bottom of an in-text float (12.0pt plus 2.0pt minus 2.0pt).
%    \dbltextfloatsep is \textfloatsep for 2 column output (20.0pt plus 2.0pt minus 4.0pt).
%    \dblfloatsep is \floatsep for 2 column output (12.0pt plus 2.0pt minus 2.0pt).
%    \abovecaptionskip: space above caption (10.0pt).
%    \belowcaptionskip: space below caption (0.0pt).
%/\/\/\/\/\/\/\/\/\/\/\/\/\/\/\/\/\/\/\/\/\/\/\/\/\/\/\/\/\/\/\/\/\/\/\/\
%					Float-Spacing done
%------------------------------------------------------------------------
%________________________________________________________________________
%
%
%
%
%
%
%
%
%
%________________________________________________________________________
%------------------------------------------------------------------------
%							Setups für Microtype
%/\/\/\/\/\/\/\/\/\/\/\/\/\/\/\/\/\/\/\/\/\/\/\/\/\/\/\/\/\/\/\/\/\/\/\/\
\SetProtrusion{encoding={*},family={bch},series={*},size={6,7}}
              {1={ ,750},2={ ,500},3={ ,500},4={ ,500},5={ ,500},
               6={ ,500},7={ ,600},8={ ,500},9={ ,500},0={ ,500}}
\SetExtraKerning[unit=space]
    {encoding={*}, family={bch}, series={*}, size={footnotesize,small,normalsize}}
    {\textendash={400,400}, % en-dash, add more space around it
     "28={ ,150}, % left bracket, add space from right
     "29={150, }, % right bracket, add space from left
     \textquotedblleft={ ,150}, % left quotation mark, space from right
     \textquotedblright={150, }} % right quotation mark, space from left
\SetTracking{encoding={*}, shape=sc}{40}
%/\/\/\/\/\/\/\/\/\/\/\/\/\/\/\/\/\/\/\/\/\/\/\/\/\/\/\/\/\/\/\/\/\/\/\/\
%							Microtype fertig
%------------------------------------------------------------------------
%________________________________________________________________________
%
%
%
%
%
%
%________________________________________________________________________
%------------------------------------------------------------------------
%						Setups für Hyperref/URL
%		Note: Package hyperref internally laods package url
%		(Like some other Packages do. For Example: biblatex)
%/\/\/\/\/\/\/\/\/\/\/\/\/\/\/\/\/\/\/\/\/\/\/\/\/\/\/\/\/\/\/\/\/\/\/\/\
%\mathchardef\UrlBreakPenalty=100% I think default is 100
% Allow linebreaks in URLs after additional character to the defaults, but
% don't just \renew or \def it, because this would remove all characters
% predefined within \UrlBreaks in the package. This could mislead other users.
% E.g. can cause smaller dots.. So use the etoolbox with "append to"
%\appto\UrlBreaks{%
%	\do\*%
% 	\do\-% Is done with the url option "hyphens"
%	\do\~%
%	\do\"%
%	\do\a\do\b\do\c\do\d\do\e\do\f\do\g\do\h\do\i\do\j%
%	\do\k\do\l\do\m\do\n\do\o\do\p\do\q\do\r\do\s\do\t\do\u\do\v\do\w%
%	\do\x\do\y\do\z%
%}%
%	Note: Alread inside the \UrlBreaks by default are:
% 	\do\.%
% 	\do\=%
% 	\do\'%
% 	\do\&%
% 	\do\-% With the option hypens
% \expandafter\def\expandafter\UrlBreaks\expandafter{\UrlBreaks% save the current one
% 	\do\*%
% 	\do\-%
% 	\do\~%
% 	\do\'%
% 	\do\"%
% 	\do\-%
% 	\do\&%
% 	\do\=%
% }%
%/\/\/\/\/\/\/\/\/\/\/\/\/\/\/\/\/\/\/\/\/\/\/\/\/\/\/\/\/\/\/\/\/\/\/\/\
%						Hyperref/URL fertig
%------------------------------------------------------------------------
%________________________________________________________________________
%
%
%
%
%
%
%
%________________________________________________________________________
%------------------------------------------------------------------------
%						Redefinition of \texttt
%				to allow a proper hyphenation (Silbentrennung)
% other breakpoint symbols can be added, for example, below I also make [ a breakpoint:
%/\/\/\/\/\/\/\/\/\/\/\/\/\/\/\/\/\/\/\/\/\/\/\/\/\/\/\/\/\/\/\/\/\/\/\/\
%\let\stdtexttt\texttt
% \newcommand{\newtexttt}[1]{%
%   \begingroup
%   \ttfamily
%   \begingroup\lccode`~=`/\lowercase{\endgroup\def~}{/\discretionary{}{}{}}%
%   \begingroup\lccode`~=`[\lowercase{\endgroup\def~}{[\discretionary{}{}{}}%
%   \begingroup\lccode`~=`.\lowercase{\endgroup\def~}{.\discretionary{}{}{}}%
%   \catcode`/=\active\catcode`[=\active\catcode`.=\active
%   \scantokens{#1\noexpand}%
%   \endgroup
% }
% Oh noes, better way: Look at the Schriftart Segment in the Header
%/\/\/\/\/\/\/\/\/\/\/\/\/\/\/\/\/\/\/\/\/\/\/\/\/\/\/\/\/\/\/\/\/\/\/\/\
%						Redef fertig
%------------------------------------------------------------------------
%________________________________________________________________________
%
%
%
%
%
%________________________________________________________________________
%------------------------------------------------------------------------
%						Definitionen allgemeiner Farben
%/\/\/\/\/\/\/\/\/\/\/\/\/\/\/\/\/\/\/\/\/\/\/\/\/\/\/\/\/\/\/\/\/\/\/\/\
%\definecolor{optionalargscolor}{RGB}{255,125,25}%
%/\/\/\/\/\/\/\/\/\/\/\/\/\/\/\/\/\/\/\/\/\/\/\/\/\/\/\/\/\/\/\/\/\/\/\/\
%							Farben fertig
%------------------------------------------------------------------------
%________________________________________________________________________
%
%
%
%
%
%
%
%
%
%
% Initialisierungen wie Counter, Theorem, Variablen, Konstanten..
%
%\theoremstyle{break}
%\theoremheaderfont{\bfseries}
%\theorembodyfont{\normalfont}
%\theoremseparator{}
%\theorempreskip{0ex}
%\theorempostskip{0ex}
%\theoremindent0.5em
%
%
% 
% 
%
%
\input{"\DenKrLayoutMainRootDir/3settings/listings_listings".tex}%
% \input{"\DenKrLayoutMainRootDir/3settings/listings_minted".tex}%