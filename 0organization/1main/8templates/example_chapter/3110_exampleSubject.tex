
% \tikzabb%[tex]%
% {tikztest}%
% {%
% Tikz-Picture Caption.%
% }%
% {1}[1,1,!ht]%[0.55,1,!ht]%
% {fig:tikz_test}%



\section{Some Example Section Name}

Example-Citation:
\cite{DenKr_denkrement1_indeco}
% \nl%
% A custom citation command that puts information about the reference to the footnote on first occurrence:
% \citeff{DenKr_MigArb,DenKr_denkrement1_indeco}%

\npi%
Some special characters:
»«.
\nl%
\gerguiquote{Done with Macro}
\nl%
Example Acronym\nl
\gls{fsa}

\npi%
Document-within References to check whether hyperref \& nameref work properly:\nl%
\ref{sec:intro} (\nameref{sec:intro}). \ref{sec:RelWork}  (\nameref{sec:RelWork}). \ref{sec:Concl} (\nameref{sec:Concl})\nl%
Consider using my Macros:\nl
\eleref{fig:Pic} or \elenumnamref{tab:OI_app}

\npi%
\begin{itemize}
\item%
    Example Itemization
\end{itemize}
\begin{enumerate}
\item%
    Example Enumeration
\end{enumerate}
\begin{description}
\item[Example Description List]%
    Just some random Text without any sense, but with sufficient length to cause a line-break, so that the indentation is properly showing influence.
\end{description}
\begin{enuminlrom}
\item%
    Example Inline Roman Enumeration
\item%
    Item 2.
\end{enuminlrom}

\np
\newcommand{\I}{\mathrm{i}}
$y = \int_0^x\cos(x)\,\mathrm{d}{x} = \frac{e^{\I x} - e^{-\I x}}{2\I} | 0123456789$
\nl
\begin{equation}
y = \int_0^x\cos(x)\,\mathrm{d}{x} = \frac{e^{\I x} - e^{-\I x}}{2\I} | 0123456789
\end{equation}

\npi%
{%
    \LARGE%
    \contourlength{\DenKrOutlineWidth}% The Default-Value here
    \contour{violet}{\textcolor{orange}{Text with Outline/Contour}}\nl%
    {%
        \contourlength{0.2em}%
        \contour{violet}{\textcolor{orange}{Text with Outline/Contour}}\nl%
        \contour{violet}{\textcolor{orange}{Text with Outline/Contour}}\nl%
    }%
    \contour{violet}{\textcolor{orange}{Text with Outline/Contour}}\nl%
}%



\nl%
%
\begin{figure}[!htpb]
    \centering
	\includestandalone[mode=\includestandalonedefaultmode]{\tikzFilesPath/tikztest}% width=\columnwidth
    \caption{Tikz-Picture Caption. Some Example Standalone-Tikz-Pic. Examples for Standalone-TikZ Picture Files can be found in \enquote{0organization/1main/8templates/tikz/7Tikz}.}
    \label{fig:tikz_test}
\end{figure}
%
%
% \tikzabb%[tex]%
% {tikztest}%
% [\DenKrLayoutMainRootDir/8templates/tikz/7Tikz]% Alternative Path to the std "./7Tikz"
% {%
% Tikz-Picture Caption. Some Example Standalone-Tikz-Pic
% }%
% {1}[1,1,!ht]%[0.55,1,!ht]%
% {fig:tikz_test2}%
%
%
\begin{figure}%[!ht]
\centering
    \includegraphics[width=0.3\linewidth]{{"\DenKrGraphicsRootDir/example_AGV"}.pdf}%
	\caption{Some another Pic}%
	\label{fig:Pic}
\end{figure}
%
%
\lstinputlisting[
    frame=single,
    label=lst:waitingChain,
    caption={Example-Listing (for Programming-Language \textit{C})},
    captionpos=b,
    language=DenKr-C,
    morekeywords={[3]{
        some_struct
    }},
    morekeywords={[4]{
        entry,
        member,
        iterator
    }}
]
{\DenKrListingsRootDir/example_lst_C.c}
%
%
\lstinputlisting[
    frame=single,
    label=lst:cfgCalc,
    caption={Example-Code-Listing (for Pseudo-Code)},
    captionpos=b,
    language=DenKr-JavaScript
]
{\DenKrListingsRootDir/example_lst.pseudo}
%
%
\input{"\DenKrTablesRootDir/example_table".tex}%
%
%